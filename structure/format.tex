% This file holds various formatting definitions

% ~~~~~~~~~~~~~~~~~~~~
% MARGINS
% best not to touch these...
\setlength{\marginparwidth}{50pt}
\setlength{\marginparsep}{10pt}
\setlength{\marginparpush}{5pt}
\setlength{\topmargin}{5pt}
\setlength{\headheight}{15pt}
\setlength{\headsep}{25pt}
\setlength{\footskip}{30pt}
\setlength{\textwidth}{\paperwidth}
\addtolength{\textwidth}{-\marginparsep}
\addtolength{\textwidth}{-\oddsidemargin}
\addtolength{\textwidth}{-\marginparwidth}
\addtolength{\textwidth}{-1in}
\addtolength{\textwidth}{-1mm}
\addtolength{\textwidth}{-\hoffset}
\setlength{\textheight}{\paperheight}
\addtolength{\textheight}{-\topmargin}
\addtolength{\textheight}{-\headheight}
\addtolength{\textheight}{-\headsep}
\addtolength{\textheight}{-\footskip}
\addtolength{\textheight}{-1in}
\addtolength{\textheight}{-\voffset}
\addtolength{\textheight}{-2cm}
\setlength{\columnsep}{10pt}
\setlength{\columnseprule}{0pt}
\setlength{\LTcapwidth}{8in}

% ~~~~~~~~~~~~~~~~~~~~
% LINE SPACING
% a thesis is usually double spaced throughout, it makes it easier to add comments
% though you could use 1.5 spacing if you'd like
\doublespacing
%\onehalfspacing

% ~~~~~~~~~~~~~~~~~~~~
% HEADINGS
% here we set the text colours for headings (see colours.txt for definition)
\chapterfont{\color{SheffieldPurple}}
\sectionfont{\color{SheffieldPurple}}
\subsectionfont{\color{SheffieldPurple}}
\subsubsectionfont{\color{SheffieldPurple}}
% We also clear the default lables for the contents sections, so we can chose our own (see support\contents.tex)

% ~~~~~~~~~~~~~~~~~~~~
% HEADERS & FOOTERS
% there are three page formats we use throughout the document:
% - \pagestyle{empty} - no headers or footers (e.g. the title page)
% - \pagestyle{plain} - no header, page number centred in the footer (front/back sections e.g. contents, bibliography)
% - \pagestyle{fancy} - chapter/section names in the header, no footer (normal pages i.e. most of the thesis)
% you can see we swap between them in the main thesis.tex file

% here we define the "fancy" format used for most of the content pages
% first overwrite the default commands, because they are ugly
\renewcommand{\chaptermark}[1]{\markboth{#1}{}}
\renewcommand{\sectionmark}[1]{\markright{#1}}
% ---------
% for onepage formatting (the default) uncomment these two lines, and comment out the three below:
\fancyhead[L]{\chaptername\ \thechapter: \leftmark}
\fancyhead[R]{\textbf{\thepage}}
% ---------
% for twopage formatting (when printing) uncomment these three lines, and comment out the two above:
%\fancyhead[LE, RO]{\textbf{\thepage}}
%\makeatletter\fancyhead[RE]{\@chapapp\ \thechapter: \leftmark}\makeatother
%\fancyhead[LO]{Section \thesection: \rightmark}
% ---------
% in both one and two page modes the fotter is just blank
\fancyfoot{}