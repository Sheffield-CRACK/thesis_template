% Sheffield Astronomy PhD Thesis Template
%
% This is the main file that produces the output PDF.
%
% Almost all of the actual content is split into individual files and subfolders to keep things organised:
% Chapters:
% - These files contain the main section of the thesis, including the introduction, conclusion and any appendixes.
% Images:
% - This folder is a place to store any images referenced to in the rest of the thesis (see the examples).
% Support:
% - These are the various other parts of the thesis, mostly front matter like the title page, summary, declaration etc.
% - This also includes files for the table of contents, bibliography etc which will be filled out automatically.
% Structure:
% - The files in here contain various LaTeX commands and functions, e.g. imported packages and formatting config.
% - Note below these are loaded using the \input{} command before the document begins, unlike the other pages
%   which use \include{}.

% ~~~~~~~~~~~~~~~~~~~~

% Document class
% You'll want to change this based on how you are going to use the PDF:
% - Use "oneside" whenever producing a digital PDF, for reading or uploading to White Rose / ArXiv
%   This is the default, you should use when submitting the thesis online
%   This will prevent the margin and page number alternating between pages, which is annoying when scrolling a PDF
\documentclass[a4paper,12pt,oneside,openright]{report}

% - Use "twoside" only when printing on actual paper
%   This will mean that the inner margin is mirrored correctly, so there's more space when its bound
%\documentclass[a4paper,12pt,twoside,openright]{report}

% NOTE: to get the headers formatted correctly for two sided layout go to `structure/format.txt` and uncomment the lines
% in the headers/footers section

% ~~~~~~~~~~~~~~~~~~~~

% Preamble structure, split off into separate files so this main file is nice and tidy

% First import any LaTeX packages to enable extra commands and features
% This file imports all the LaTex packages used in this document
% Note this must be the very first input included in the main thesis.tex file, right after the documentclass
% Some packages include configuration options when they are imported, see hyperref for an example

% ~~~~~~~~~~~~~~~~~~~~
% CORE PACKAGES

% Standard package for selecting font encodings
% https://ctan.org/pkg/fontenc
\usepackage[T1]{fontenc}

% Language support
% https://ctan.org/pkg/babel
\usepackage[UKenglish]{babel}

% Tune the output format of dates according to language
% https://ctan.org/pkg/isodate
\usepackage[UKenglish]{isodate}

% Enhanced support for graphics
% https://ctan.org/pkg/graphicx
\usepackage{graphicx}
%\usepackage[draft]{graphicx}
% NOTE: you can use the [draft] option to hide figures, handy to reduce file size (e.g. for Turnitin)

% Driver-independent colour extensions
% https://ctan.org/pkg/xcolor
\usepackage[dvipsnames]{xcolor}

% Selective filtering of error messages and warnings
% https://ctan.org/pkg/silence
\usepackage{silence}

% ~~~~~~~~~~~~~~~~~~~~
% PAGE FORMATTING

% Flexible and complete interface to document dimensions
% https://ctan.org/pkg/geometry
\usepackage[lmargin=3.0cm,rmargin=2.3cm]{geometry}

% Draw a page-layout diagram
% https://ctan.org/pkg/showframe
%\usepackage{showframe}
%\renewcommand*\ShowFrameColor{\color{red!10}}
% NOTE: this is useful if you are having issues with margins, just uncomment the above two lines

% Intermix single and multiple columns
% https://ctan.org/pkg/multicol
\usepackage{multicol}

% Set space between lines
% https://ctan.org/pkg/setspace
\usepackage{setspace}

% Control page headers and footers
% https://ctan.org/pkg/fancyhdr
\usepackage{fancyhdr}

% A range of footnote options
% https://ctan.org/pkg/footmisc
\usepackage{footmisc}

% Select alternative section titles
% https://ctan.org/pkg/titlesec
\usepackage{titlesec,titletoc}

% Add epigraphs (quotes at the start of chapters)
% https://ctan.org/pkg/epigraph
\usepackage{epigraph}

% Insert whole PDF documents with the \includepdf command
% https://ctan.org/pkg/pdfpages
\usepackage{pdfpages}

% Make a page landscape orientated
% https://ctan.org/pkg/pdflscape
\usepackage{pdflscape}

% Execute command after the next page break
% https://ctan.org/pkg/afterpage
\usepackage{afterpage}

% ~~~~~~~~~~~~~~~~~~~~
% TEXT FORMATTING

% Format headers (e.g. \sectionfont)
% https://ctan.org/pkg/sectsty
\usepackage{sectsty}

% Format hyperlinks (\href, \url)
% https://ctan.org/pkg/hyperref
\usepackage[unicode,
			breaklinks=true,  % Allow links to break over lines
			colorlinks=true,  % Colours links instead of the default boxes
            % Link colours
			linkcolor=SheffieldPurple,  % Colour of internal links (see colours.txt for definition)
			urlcolor=SheffieldPurple,  % Colour of external hyperlinks
			citecolor=SheffieldPurple,  % Colour of citations
			% PDF metadata
			pdfusetitle,      % Gets PDF metadata from \author and \title
			pdfsubject={PhD Thesis},
			pdfkeywords={astronomy}
			]{hyperref}

% A new bookmark (outline) organization for hyperref
% https://ctan.org/pkg/bookmark
\usepackage{bookmark}

% Make reference to section names, etc
% https://ctan.org/pkg/nameref
\usepackage{nameref}

% Context sensitive quotation facilities
% https://ctan.org/pkg/csquotes
\usepackage{csquotes}

% ~~~~~~~~~~~~~~~~~~~~
% SECTIONS

% Flexible bibliography support
% https://ctan.org/pkg/natbib
\usepackage{natbib}

% Extra control of appendices
% https://ctan.org/pkg/appendix
\usepackage[titletoc]{appendix}

% Control table of contents, figures, etc
% https://ctan.org/pkg/tocloft
\usepackage{tocloft}

% Create glossaries and lists of acronyms
% https://ctan.org/pkg/glossaries
\usepackage[acronym,shortcuts]{glossaries}

% Produce a table of contents for each chapter, part or section
% https://ctan.org/pkg/minitoc
\usepackage{minitoc}

% ~~~~~~~~~~~~~~~~~~~~
% FIGURES

% Improved interface for floating objects (figures, tables etc)
% https://ctan.org/pkg/float
\usepackage{float}

% Extending the array and tabular environments
% https://ctan.org/pkg/array
\usepackage{array}

% Customising captions in floating environments
% https://ctan.org/pkg/caption
%\WarningFilter{caption}{\caption}
\usepackage[style=base, labelfont={bf, color=SheffieldPurple}]{caption}

% Retain float number across several floats
% https://ctan.org/pkg/captcont
%\usepackage{captcont}

% Control float placement
% https://ctan.org/pkg/placeins
\usepackage[section]{placeins}

% Rotation tools, including rotated full-page floats
% https://ctan.org/pkg/rotating
\usepackage{rotating}

% Rotate floats, i.e. combining the rotating and float packages
% https://ctan.org/pkg/rotfloat
\usepackage{rotfloat}

% Permit footnotes in tables (with \tablefootnote{})
% https://ctan.org/pkg/tablefootnote
\usepackage{tablefootnote}

% Produces figures which text can flow around
% https://ctan.org/pkg/wrapfig
\usepackage{wrapfig}

% Publication quality tables in LaTeX
% https://ctan.org/pkg/booktabs
\usepackage{booktabs}

% Allow tables to flow over page boundaries
% https://ctan.org/pkg/longtable
\usepackage{longtable}

% Create tabular cells spanning multiple rows
% https://ctan.org/pkg/multirow
\usepackage{multirow}

% Tabular column heads and multi-lined cells
% https://ctan.org/pkg/makecell
\usepackage{makecell}

% ~~~~~~~~~~~~~~~~~~~~
% SCIENTIFIC

% AMS mathematical facilities
% https://ctan.org/pkg/amsmath
\usepackage{amsmath,amssymb,amsthm}

% Comprehensive SI units and number formatting (e.g. \SI{9.81}{\meter\per\second\squared})
% https://ctan.org/pkg/siunitx
\usepackage{siunitx}
\sisetup{separate-uncertainty}  % see https://tex.stackexchange.com/questions/409323/aligning-sign-in-table

% Typeset in-line fractions in a "nice" way
% https://ctan.org/pkg/nicefrac
\usepackage{nicefrac}

% Typeset source code listings
% https://ctan.org/pkg/listings
\usepackage{listings}

% ~~~~~~~~~~~~~~~~~~~~
% SYMBOLS

% Access to PostScript standard Symbol and Dingbats fonts
% https://ctan.org/pkg/pifont
\usepackage{pifont}

% St Mary Road symbols for theoretical computer science
% https://ctan.org/pkg/stmaryrd
\usepackage{stmaryrd}

% Martin Vogel's Symbols (contains astronomical/astrological icons)
% https://ctan.org/pkg/marvosym
\usepackage{marvosym}

% Split-level fractions (for small in-text fractions, use \sfrac{1}{2})
% https://ctan.org/pkg/xfrac
\usepackage{xfrac}

% ~~~~~~~~~~~~~~~~~~~~
% MISCELLANEOUS
% Add \verbatim and \comment environments
% https://ctan.org/pkg/verbatim
\usepackage{verbatim}

% Formats for dates, times and time zones
% https://ctan.org/pkg/datetime2
\usepackage{datetime2}

% Easy access to the Lorem Ipsum dummy text (\lipsum)
% https://ctan.org/pkg/lipsum
\usepackage{lipsum}

% Mark things to do
% https://ctan.org/pkg/todonotes
%\usepackage[textsize=footnotesize]{todonotes}

% Extended conditional commands
% https://ctan.org/pkg/xifthen
\usepackage{xifthen}

% Extending etoolbox patching commands
% https://ctan.org/pkg/xpatch
\usepackage{xpatch}

% This file defines the page formats (e.g. margins, headers)
% This file holds various formatting definitions

% ~~~~~~~~~~~~~~~~~~~~
% MARGINS
% best not to touch these...
\setlength{\marginparwidth}{50pt}
\setlength{\marginparsep}{10pt}
\setlength{\marginparpush}{5pt}
\setlength{\topmargin}{5pt}
\setlength{\headheight}{15pt}
\setlength{\headsep}{25pt}
\setlength{\footskip}{30pt}
\setlength{\textwidth}{\paperwidth}
\addtolength{\textwidth}{-\marginparsep}
\addtolength{\textwidth}{-\oddsidemargin}
\addtolength{\textwidth}{-\marginparwidth}
\addtolength{\textwidth}{-1in}
\addtolength{\textwidth}{-1mm}
\addtolength{\textwidth}{-\hoffset}
\setlength{\textheight}{\paperheight}
\addtolength{\textheight}{-\topmargin}
\addtolength{\textheight}{-\headheight}
\addtolength{\textheight}{-\headsep}
\addtolength{\textheight}{-\footskip}
\addtolength{\textheight}{-1in}
\addtolength{\textheight}{-\voffset}
\addtolength{\textheight}{-2cm}
\setlength{\columnsep}{10pt}
\setlength{\columnseprule}{0pt}
\setlength{\LTcapwidth}{8in}

% ~~~~~~~~~~~~~~~~~~~~
% LINE SPACING
% a thesis is usually double spaced throughout, it makes it easier to add comments
% though you could use 1.5 spacing if you'd like
\doublespacing
%\onehalfspacing

% ~~~~~~~~~~~~~~~~~~~~
% HEADINGS
% here we set the text colours for headings (see colours.txt for definition)
\chapterfont{\color{SheffieldPurple}}
\sectionfont{\color{SheffieldPurple}}
\subsectionfont{\color{SheffieldPurple}}
\subsubsectionfont{\color{SheffieldPurple}}
% We also clear the default lables for the contents sections, so we can chose our own (see support\contents.tex)

% ~~~~~~~~~~~~~~~~~~~~
% HEADERS & FOOTERS
% there are three page formats we use throughout the document:
% - \pagestyle{empty} - no headers or footers (e.g. the title page)
% - \pagestyle{plain} - no header, page number centred in the footer (front/back sections e.g. contents, bibliography)
% - \pagestyle{fancy} - chapter/section names in the header, no footer (normal pages i.e. most of the thesis)
% you can see we swap between them in the main thesis.tex file

% here we define the "fancy" format used for most of the content pages
% first overwrite the default commands, because they are ugly
\renewcommand{\chaptermark}[1]{\markboth{#1}{}}
\renewcommand{\sectionmark}[1]{\markright{#1}}
% ---------
% for onepage formatting (the default) uncomment these two lines, and comment out the three below:
\fancyhead[L]{\chaptername\ \thechapter: \leftmark}
\fancyhead[R]{\textbf{\thepage}}
% ---------
% for twopage formatting (when printing) uncomment these three lines, and comment out the two above:
%\fancyhead[LE, RO]{\textbf{\thepage}}
%\makeatletter\fancyhead[RE]{\@chapapp\ \thechapter: \leftmark}\makeatother
%\fancyhead[LO]{Section \thesection: \rightmark}
% ---------
% in both one and two page modes the fotter is just blank
\fancyfoot{}
% This page defines custom colours so you can make the thesis pretty
% This file defines colours to use for section headings, links etc

% Colours are used in various places. The main ones are:
% - section heading colours, defined under "heading colours" in format.txt
% - link colours, defined in the "hyperref" package configuration (see packages.txt)
% - caption colours, defined in the "caption" package configuration (see packages.txt)

% If you just want to make coloured text use "\textcolor{red}{THIS TEXT IS RED}"

% The "xcolors" package includes a lot of default colours, see the documentation
% at https://ctan.org/pkg/xcolor for a full list.
% Here we define various custom colours, for formatting or just in case they are useful.

% Example colours, can used for headers
\definecolor{StevanceBlue}{RGB}{46,116,181}
\definecolor{SheffieldPurple}{RGB}{68,0,153} % < this is the current default, based on the Uni logo (https://www.sheffield.ac.uk/brand-toolkit/colour)

% Tableau colours (matplotlib defaults)
% https://www.tableau.com/about/blog/2016/7/colors-upgrade-tableau-10-56782
\definecolor{tab_blue}{HTML}{4E79A7}
\definecolor{tab_orange}{HTML}{F28E2B}
\definecolor{tab_red}{HTML}{E15759}
\definecolor{tab_cyan}{HTML}{76B7B2}
\definecolor{tab_green}{HTML}{59A14F}
\definecolor{tab_yellow}{HTML}{EDC948}
\definecolor{tab_purple}{HTML}{B07AA1}
\definecolor{tab_pink}{HTML}{FF9DA7}
\definecolor{tab_brown}{HTML}{9C755F}
\definecolor{tab_grey}{HTML}{BAB0AC}
% This file defines custom astronomical units for use with the siunitx package
% This file defines extra units for astronomy which can be used with the siunitx package
% e.g. \SI{100}{\solarmass\per\second}

% ~~~~~~~~~~~~~~~~~~~~
\DeclareSIUnit\parsec{pc}
\DeclareSIUnit\lightyear{ly}
\DeclareSIUnit\arcmin{arcmin}
\DeclareSIUnit\arcsec{arcsec}
\DeclareSIUnit\pixel{pixel}
\DeclareSIUnit\photon{photons}
\DeclareSIUnit\mag{mag}
\DeclareSIUnit\erg{erg}
\DeclareSIUnit\jansky{Jy}
\DeclareSIUnit\solarmass{\ensuremath{\textup{M}_\odot}}
\DeclareSIUnit\solarlum{\ensuremath{\textup{L}_\odot}}
\DeclareSIUnit\solarrad{\ensuremath{\textup{R}_\odot}}
\DeclareSIUnit\electron{\ensuremath{\textup{e}^-}}

% This file defines standard abbreviations for astronomy journals
\input{structure/journals}
% This file defines acronyms to be used using the glossaries package
\makeglossaries{}

% set the length of the glossary description on the acronym page
\setlength{\glsdescwidth}{0.8\linewidth}

% define \acro commands
% using \acro will automatically add the full name when first referenced, then use the short version on later calls
% if you want to force the full name with the acronym in brackets then use \acrofull
\newcommand{\acro}{\gls}
\newcommand{\acrofull}{\glsfirst}

% if you want hyperlinks from each reference to the acronym list then comment out this line
%\glsdisablehyper{} % no hyperlinks

% ~~~~~~~~~~~~~~~~~~~~
% Add any acronyms below this point:
\newacronym{goto}{GOTO}{Gravitational-wave Optical Transient Observer}

% This file defines other custom commands and definitions
% This file holds various LaTeX commands and definitions

% icons & symbols
\newcommand{\tick}{\ding{52}}
\newcommand{\uptriangle}{\ding{115}}
\newcommand{\backarrow}{\rotatebox[origin=c]{180}{\ding{252}}}
\newcommand{\blackdiamond}{\ding{117}}
\newcommand{\about}{\ensuremath{\sim}}
\newcommand{\done}{\textcolor{green}{~\tick~}}

% scientific symbols
\newcommand{\elec}{\ensuremath{\textup{e}^-}}

% superscripts
\newcommand{\st}{\ensuremath{^{\rm st}}}
\newcommand{\nd}{\ensuremath{^{\rm nd}}}
\newcommand{\rd}{\ensuremath{^{\rm rd}}}
\newcommand{\nth}{\ensuremath{^{\rm th}}}


% easy references to the basic properties for the titlepage
\makeatletter
\let\thetitle\@title
\let\theauthor\@author
\let\thedate\@date
\makeatother

% reference functions containing the type, also name
\AtBeginDocument{%
    \renewcommand\chapterautorefname{Chapter}%
    \renewcommand\sectionautorefname{Section}%
    \renewcommand\subsectionautorefname{Section}%
}
\newcommand\aref[1]{\autoref{#1}}
\newcommand\nref[1]{\autoref{#1} (\nameref{#1})}


\DeclareMathOperator{\sech}{sech}
\newcommand{\Msun}{\ensuremath{\textrm{\,M}_{\odot}}}
\newcommand{\Lsun}{\ensuremath{\textrm{\,L}_{\odot}}}
\newcommand{\Mbh}{\ensuremath{\textrm{\,M}_{BH}~}}
\newcommand{\Mbu}{\ensuremath{\textrm{\,M}_{bulge}~}}
\newcommand{\rat}{\ensuremath{L/L_{edd}}~}
\newcommand{\degs}{\ensuremath{^{\circ}}}
\newcommand{\kms}{\ensuremath{\textrm{\,km s}^{-1}}}
\newcommand{\chisq}{\ensuremath{\chi^{2}_{red}}}
\newcommand{\chisqlt}{\ensuremath{\chi^{2}_{red} < 1}}
\newcommand{\ergs}{\ensuremath{\textrm{\,erg s}^{-1}}}
\newcommand{\microns}{\ensuremath{\mu{\mbox{m}}}}
\newcommand{\hb}{\ensuremath{\mbox{H}{\beta}}}
\newcommand{\hg}{\ensuremath{\mbox{H}{\gamma}}}
\newcommand{\hd}{\ensuremath{\mbox{H}{\delta}}}
\newcommand{\ha}{\ensuremath{\mbox{H}{\alpha}}}
\newcommand{\ebv}{E(B--V)}
\newcommand{\rv}{\ensuremath{\mathrm{R_V}}}
\newcommand{\ca}{Ca {\sc ii} K}
\newcommand{\ew}{\ensuremath{\mathrm{W}_\lambda}}
\newcommand{\oiii}{\ensuremath{\mathrm{[OIII]}\,88.36\mu\mathrm{m}}}
\newcommand{\otwop}{\ensuremath{\mathrm{O}^{2+}}}
\newcommand{\othrp}{\ensuremath{\mathrm{O}^{3+}}}
\newcommand{\netwop}{\ensuremath{\mathrm{Ne}^{2+}}}
\newcommand{\nep}{\ensuremath{\mathrm{Ne}^{+}}}
\newcommand{\stwop}{\ensuremath{\mathrm{S}^{2+}}}
\newcommand{\sthrp}{\ensuremath{\mathrm{S}^{3+}}}

\newcommand{\ds}{\ensuremath{\mathrm{d}s}}
\newcommand{\dt}{\ensuremath{\mathrm{d}t}}
\newcommand{\dx}{\ensuremath{\mathrm{d}x}}
\newcommand{\dy}{\ensuremath{\mathrm{d}y}}
\newcommand{\dz}{\ensuremath{\mathrm{d}z}}

\newcommand{\msol} {M$_{\odot}$}
\newcommand{\lsol} {L$_{\odot}$}
\newcommand{\mza} {M$_{ZAMS}$}
\newcommand{\zsol} {Z$_{\odot}$}
\newcommand{\rsol} {R$_{\odot}$}
\newcommand{\metal}{12 + log(O/H) }
%\newcommand\ion[2]{#1$\;${\small\rmfamily\@Roman{#2}}}
%\newcommand\ion[2]{#1$\;${\scshape{#2}}}% 
\newcommand{\electron} {$\mathrm{e^{-}}$}
\def\lesssim{\mathrel{\hbox{\rlap{\hbox{\lower4pt\hbox{$\sim$}}}\hbox{$<$}}}}
\def\gtrsim{\mathrel{\hbox{\rlap{\hbox{\lower4pt\hbox{$\sim$}}}\hbox{$>$}}}}

\newcommand{\degree}{$^{\circ}$}
%\newcommand{\arcsec}{\mbox{$^{\prime\prime}$}}%
\newcommand{\deps}{$\Delta \epsilon\,$}

\newcommand{\halpha} {$\mathrm{H\alpha}$ }
\newcommand{\hbeta} {$\mathrm{H\beta}\,$}
%\newcommand{\heI5876} {HeI $\lambda 5876$}
%\newcommand{\heI6678} {HeI $\lambda 6678$}
%\newcommand{\heI7065} {HeI $\lambda 7065$}

\long\def\symbolfootnote[#1]#2{\begingroup%
\def\thefootnote{\fnsymbol{footnote}}\footnote[#1]{#2}\endgroup} 



% ~~~~~~~~~~~~~~~~~~~~

\begin{document}

% Basic document info, this is used on the title page and in the PDF metadata
\title{An Example PhD Thesis in Astronomy}
\author{Albert N. Example}
\date{\today}  % replace with your actual submission date (e.g. \date{1st January 2000})

% ~~~~~~~~~~~~~~~~~~~~

% Now we have the actual content of the thesis
% Each section/chapter is split into a separate file to keep things tidy,
% with the main text in "chapters" and the rest in "support".
% Different parts of the thesis have different formatting styles, these are defined
% below before the content files are included.

% Cover pages
% From the start we use Roman numerals (i, ii, etc) until we get to the first chapter.
% However the very first few pages have no visible page numbers, so we use the "empty" style.
\pagenumbering{roman}
\pagestyle{empty}
% ---------
% This file defines the thesis title page

% The format is defined by the university, it must include:
% - The full title of the thesis.
% - The author's name in full.
% - The degree for which the thesis is submitted.
% - The Department in which the work has been carried out.
% - The date (month and year) of submission.

% Otherwise the exact layout isn't fixed.
% If you want to include a different logo (e.g. the coat of arms) then that's up to you,
% there are some examples in the /images directory

% Note the actual title, name and date are defined in the main `thesis.tex` file,
% right after \begin{document}.

% ~~~~~~~~~~~~~~~~~~~~
\begin{center}
    \vspace*{2cm}

    \begin{Huge}
        \makeatletter
        \textbf{\@title}
        \makeatother
    \end{Huge}

    \vspace*{2.5cm}

    \begin{LARGE}
        \makeatletter
        \textbf{\@author}
        \makeatother
    \end{LARGE}

    \vspace*{1cm}

    \begin{Large}
        Department of Physics and Astronomy \\
        \smallskip
        The University of Sheffield
    \end{Large}

    \vspace*{2cm}

    \includegraphics[width=0.5\linewidth]{images/sheffield_newlogo.jpg}

    \vspace*{2cm}

    \begin{large}
        \textit{A dissertation submitted in candidature for the degree of} \\
        \textit{Doctor of Philosophy at the University of Sheffield}
    \end{large}

    \vspace*{1cm}

    \begin{large}
        \makeatletter
        \text{\@date}
        \makeatother
    \end{large}

    \vfill
\end{center}

% This page defines the traditional quote to place at the start of your thesis.

% ~~~~~~~~~~~~~~~~~~~~
\cleardoublepage{}  % This makes sure in two-page mode the quote will be on the right

\vspace*{\fill}

\begin{singlespace}

\begin{LARGE}
    \hspace*{3.5cm}                            % < Adjust this to align nicely with the width of your quote
    \textbf{\textcolor{SheffieldPurple}{``}}
    
    \vspace*{-1.3\baselineskip}
    \begin{center}
        Put a quote here.\\
        Whatever you want,\\
        it's up to you.
    \end{center}
    \vspace*{-1.3\baselineskip}
    
    \hspace*{10.5cm}                            % < Adjust this to align nicely with the width of your quote
    \textbf{\textcolor{SheffieldPurple}{''}}
\end{LARGE}

\vspace{\baselineskip}

\begin{large}
    \hspace*{9cm} --- A. N. Example
\end{large}

\end{singlespace}

\vspace*{\fill}
% ---------

% Front matter
% Now we switch to the "plain" style, which just has a single page number at the bottom of the page.
% Note these are still Roman numerals.
\pagestyle{plain}
% ---------
\chapter*{Declaration}

% ~~~~~~~~~~~~~~~~~~~~
\begin{onehalfspace}

\noindent I declare that, unless otherwise stated, the work presented in this thesis is my own.

\bigskip

\noindent No part of this thesis has been accepted or is currently being submitted for any other qualification at the University of Sheffield or elsewhere.

\bigskip

\noindent The following chapters are based wholly or in part on previous publications:

\begin{itemize}
    \item \nref{chap:chapter1}: \citet{example}
    \item \nref{chap:chapter2}: in prep
\end{itemize}

\end{onehalfspace}
\include{support/acknowledgements}

\chapter*{Summary}

300 words max
% This cocument defines the four contents pages at the start of the thesis:
% - Table of Contents
% - List of Figures
% - List of Tables
% - List of Acronyms (see `structure/acronyms.tex`)

% We do a few clever things on top of using the default commands (e.g. \tableofcontents)
% - We use the \chapter* command to make sure the heading is in our chosen colour,
%   and so you can change it if you'd like (e.g. just "Figures" instead of "List of Figures").
%   This requires redefining the default names below (note also "title={}" in \printglossary)
\renewcommand{\contentsname}{}
\renewcommand{\listfigurename}{}
\renewcommand{\listtablename}{}
% - We also add lables and pdfbookmarks, which don't appear by default for non-numbered chapters.
% - For style we also enforce single spacing so the lines aren't too far apart,
%   and we disable the hyperlink colours so you can still use the links but it's not all purple.

% ~~~~~~~~~~~~~~~~~~~~
% CONTENTS
\chapter*{Contents}
\label{contents}
\pdfbookmark[section]{Contents}{contents}
\vspace{-3.5cm}

\begin{singlespacing}
{\hypersetup{linkcolor=black}
\tableofcontents
}
\end{singlespacing}
\clearpage

% ~~~~~~~~~~~~~~~~~~~~
% FIGURES
\chapter*{Figures}
\label{figures}
\pdfbookmark[section]{Figures}{figures}
\vspace{-3.5cm}

\begin{singlespacing}
{\hypersetup{linkcolor=black}
\listoffigures
}
\end{singlespacing}
\clearpage

% ~~~~~~~~~~~~~~~~~~~~
% TABLES
\chapter*{Tables}
\label{tables}
\pdfbookmark[section]{Tables}{tables}
\vspace{-3.5cm}

\begin{singlespacing}
{\hypersetup{linkcolor=black}
\listoftables
}
\end{singlespacing}
\clearpage

% ~~~~~~~~~~~~~~~~~~~~
% ACRONYMS
\chapter*{Acronyms and Abbreviations}
\label{acros}
\pdfbookmark[section]{Acronyms and Abbreviations}{acros}
\vspace{-3.5cm}

\begingroup
\let\clearpage\relax\vspace{-12pt}
\begin{singlespacing}
\printglossary[title={},
               type=\acronymtype,
               style=super,
               nopostdot,
               nonumberlist,
               %nogroupskip,
               ]
\end{singlespacing}
\endgroup

\clearpage  % contains Contents, Figures, Tables, and Acronyms pages
% ---------

% Main body
% Now we start the first chapter we can switch to Arabic numerals, which start again from 1.
% We also switch to the "fancy" style, which is defined in `structure/format.tex`.
\pagenumbering{arabic}
\pagestyle{fancy}
% ---------
\chapter{Introduction}
\label{chap:intro}

% ~~~~~~~~~~~~~~~~~~~~
\section{Why Astronomy is Great}
\label{sec:why}

Astronomy is all about looking up at the sky.

% ~~~~~~~~~~~~~~~~~~~~
\subsection{Examples of astronomy}
\label{sec:examples}

Some astronomy is observational and some is theoretical. There is a lot of debate on which is better.

\begin{figure}[t]
    \begin{center}
        \includegraphics[width=0.9\linewidth]{images/m31.jpg}
    \end{center}
    \caption[An image of M31]{
        This is a picture of M31, the Andromeda Galaxy.
        }\label{fig:m31}
\end{figure}

% ~~~~~~~~~~~~~~~~~~~~
\section{Thesis outline}
\label{sec:outline}

This thesis is going to show just how great astronomy is. In \nref{chap:chapter1} I describe my main findings. In \nref{chap:chapter2} I prove my theory. Finally in \nref{chap:conclusion} I present concluding remarks and some suggestions for future work.

\newpage

Here's some more test just to see if the headers are working.
\chapter{Example Chapter}
\label{chap:chapter1}
% ~~~~~~~~~~~~~~~~~~~~

Astronomy

Sheffield helped build the \acro{goto} telescope on La Palma.

There is also a second \acro{goto} site in Australia, at Siding Spring Observatory.

The full name is the \acrofull{goto}.
\chapter{Example Chapter}
\label{chap:chapter2}
% ~~~~~~~~~~~~~~~~~~~~

Did you know you can nicely format units with the \texttt{siunitx} package? For instance, \SI{100}{\solarmass\per\second} or $F_\lambda = 5.5 \times 10^{6}/\lambda_\text{eff}~\si{\photon\per\second\per\centi\metre\squared\per\angstrom}$. You use \texttt{\textbackslash{}SI\{value\}\{unit\}}, or just \texttt{\textbackslash{}si\{unit\}} if you want the unit on its own.
% ... <add your chapters here>
\chapter{Conclusion}
\label{chap:conclusion}

Here are some interesting papers: \citep{1984MNRAS.208..955F, 1991MNRAS.252..342D, 1992MolPh..75..917G, 1993PhDT.......301C}.
% ---------

% Bibliography
% Now we go back to plain style, but continuing with Arabic numerals. 
\pagestyle{plain}
% ---------
% Use the MNRAS bib style file
\bibliographystyle{mnras.bst}

% All entries are defined in the .bib file
\bibliography{bibliography.bib} 
% ---------

% Appendices
% Back to fancy style for any appendices
\pagestyle{fancy}
% We also want to change the formatting so instead of Chapter 5,6... we have Appendix A,B...
% The \appendix command takes care of all of this, including the headers, table of contents and figure labels.
% This means you can easily move chapters between the main body and appendixes without having to change
% anything within the actual chapter file.
\appendix
% ---------
\chapter{Example Appendix}
\label{appendix:example}
% ~~~~~~~~~~~~~~~~~~~~

Here is a table:

\begin{table}
	\centering
	\caption{This is an example table. Captions appear above each table.
	Remember to define the quantities, symbols and units used.}
	\label{tab:example_table}
	\begin{tabular}{lccr} % four columns, alignment for each
		\hline
		A & B & C & D\\
		\hline
		1 & 2 & 3 & 4\\
		2 & 4 & 6 & 8\\
		3 & 5 & 7 & 9\\
		\hline
	\end{tabular}
\end{table}
% ---------

% And that's it!
\end{document}